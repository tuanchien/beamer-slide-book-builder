% packages.tex
% Loads the packages required for build.

%------------------------------------------------------------------------------
% GLOBAL (load in all circumstances)
%------------------------------------------------------------------------------
\usepackage[utf8]{inputenc}
\usepackage[T1]{fontenc}
\usepackage{latexsym}
\usepackage{makeidx}
\usepackage{amsmath}
\usepackage{array}
\usepackage{latexsym}
\usepackage{amssymb}
\usepackage{amsfonts}
\usepackage{amsthm}
\usepackage[dvips]{epsfig}
\usepackage{graphicx} % Allows graphics import.
\usepackage{times}
\usepackage{float} % Float environment for figures/pictures.
\usepackage{multicol} % Used for tables.
\usepackage{multirow} % Used for tables.
\usepackage{fancybox} % Boxes and frames.
\usepackage{framed} % Frames around text, etc.
\usepackage{stackrel} % Putting things above/below other things, e.g., with arrows.
\usepackage{mathrsfs}
\usepackage{color}
\usepackage{hyperref} % Gives nice hyperlinks in notes.
\usepackage{verbatim} % Typewriter style. Used for displaying code.
\usepackage{booktabs} % Allows the use of \toprule, \midrule and \bottomrule in tables
\usepackage{caption}
\usepackage{subcaption}

%------------------------------------------------------------------------------
% NOTES ONLY
%------------------------------------------------------------------------------
\mode<article> {
\usepackage{paralist} % Gives nice list/enum control. Not strictly necessary.
\usepackage[english]{babel}
}

%------------------------------------------------------------------------------
% SLIDES ONLY
%------------------------------------------------------------------------------
\mode<presentation> {
 % Necessary for beamer slides.
}
