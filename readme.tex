\documentclass{article}
\usepackage{verbatim}

\parindent=0pt

\begin{document}

\section{Document Creation (PDF)}
\subsection{To build on a machine with linux or Mac OS X}
\begin{enumerate}
\item Open terminal and go to the project's directory
\item Type \begin{verbatim}make slides\end{verbatim} to create slides; or
\item Type \begin{verbatim}make notes\end{verbatim} to create the notes; or
\item Type \begin{verbatim}make refs\end{verbatim} to create textbook references.
\item Type \begin{verbatim}make\end{verbatim} to create slides, notes and textbook references.
\end{enumerate}

\subsection{To build using a LaTeX IDE (e.g., texniccenter or texshop)}
\begin{enumerate}
\item Build main.notes.tex (pdflatex) to get notes.
\item Build main.slides.tex (pdflatex) to get slides.
\item Build textbookrefs.tex (pdflatex) to get slides.
\end{enumerate}

\section{Maintenance of latex files}
PREREQUISITES: A working knowledge of latex and beamer.\\

Please familiarise yourself with this file hierarchy description before editing.\\

Almost all of your favourite beamer and article mode luxuries should be usable
using this build system.\\

The files are mostly organised into blocks for things affecting both notes and
slides, slides only, and notes only.\\

\subsection{File Hierarchy}
\begin{verbatim}
main.tex - chapter creation, rearrangement of content order
should happen here.

packages.tex - contains the packages to be loaded.

formatting.tex - edit the formatting rules of the notes and
slides (margins, themes, etc).

commands.tex - put all your custom \newcommands and \defs
etc in here.

environments.tex - put all your custom latex environments
in here.

variables.tex - put all your custom latex variables here.

precontent.tex - title pages, table of contents, and anything
before actual content should go here.

textbookrefs.tex - contains an independent file of textbook
references.

main.notes.tex - necessary to setup lecture note building.
Mostly leave alone.

main.slides.tex - necessary to setup slide building.
Mostly leave alone.

Makefile - responsible for the 'make' commands.
Mostly leave alone.
\end{verbatim}

\subsection{Adding chapters}
\begin{enumerate}
\item Open up main.tex

\item Add the following:
\begin{verbatim}
%---------------------------------------------------------
% YOURCHAPTERNAME
%---------------------------------------------------------
\chapter*{YOURCHAPTERNAME}
%---------------------------------------------------------
% Fix counters etc and add table of contents entry
%---------------------------------------------------------
\addtocounter{chapter}{1}
\setcounter{section}{0}
\addtocontents{toc}{{\bf YOURCHAPTERNAME}}

%---------------------------------------------------------
% Add material here
%---------------------------------------------------------
\end{verbatim}
\end{enumerate}

e.g.,
\begin{verbatim}
%---------------------------------------------------------
% CALCULUS
%---------------------------------------------------------
\chapter*{Calculus}
%---------------------------------------------------------
% Fix counters etc and add table of contents entry
%---------------------------------------------------------
\addtocounter{chapter}{1}
\setcounter{section}{0}
\addtocontents{toc}{{\bf Calculus}}

%---------------------------------------------------------
% Add material here
%---------------------------------------------------------
\end{verbatim}


\subsection{Adding topics}
\begin{enumerate}
\item Create a new file called nameoftopic.tex, e.g., hilbertspaces.tex in an
appropriate place. e.g., in the calculus subdirectory.
\item Add \begin{verbatim}\mode*\end{verbatim} so that
it is the first line.
\item Write latex as per usual afterwards.
\item Open up main.tex and find the chapter you want to add it under.  If chapter
does not exist, then see the section on adding chapters.
\item Go to the
\begin{verbatim}
%---------------------------------------------------------
% Add material here
%---------------------------------------------------------
\end{verbatim}
part of the chapter.

\item Add the following after the dotted line:
\begin{verbatim}
\mode<all> \input{/path/to/new/texfile},
\end{verbatim}

e.g.,
\begin{verbatim}
\mode<all> \input{./calculus/hilbertspaces.tex}
\end{verbatim}

in that path, the ``.'' refers to the current directory, and so
\begin{verbatim}
./calculus/
\end{verbatim} means
look in the calculus directory contained in the current directory.
\end{enumerate}

\subsection{Content: notes and slides}
By default, content from your tex files will be put into the lecture notes. If
you wish for some content to appear on slides, then you want to encapsulate it
inside a frame environment,
e.g.,
\begin{verbatim}
\begin{frame}
This text will appear in both notes and slides.
\begin{theorem}
This theorem will be in both notes and slides.
\end{theorem}
\end{frame}
\end{verbatim}

\subsection{Content: slides only}
Sometimes you want some slide only content.  You can achieve this by adding:
\begin{verbatim}
\mode<presentation> {
Stuff in here will only appear in slides.
}
\end{verbatim}

e.g.,
\begin{verbatim}
\begin{frame}
\mode<presentation> {
\frametitle{This slide title is only relevant for slides.}
}
This text will appear in both notes and slides.
\begin{theorem}
This theorem will be in both notes and slides.
\end{theorem}
\end{frame}
\end{verbatim}

\subsection{Content: notes only}
By default, unless you have a frame environment around something, it will only
appear in the notes.\\

In case you ever need to guarantee notes only things where the default may not
kick in, then use
\begin{verbatim}
\mode<article> {
stuff in here will be notes only
}
\end{verbatim}

For content, this should never need to be used.  It's more useful for editing
the control flow of the latex templates before the content.

\end{document}
